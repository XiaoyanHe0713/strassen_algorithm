\documentclass{article}
\usepackage{fullpage}
\usepackage{amsmath}
\usepackage{amssymb}

\title{Note for Strassen Algorithm}
\author{Xiaoyan He}
\date{2023-02-22}

\begin{document}

\maketitle

\section{Ideas of Basic Matrix Multiplication}
The idea behind Strassen algorithm is in the formulation
of matrix multiplication as a recursive problem.
Assume that we have two matrices in $\mathbb{R}^{n \times n}$  $A$ and $B$. We can write $A$ and $B$ as follows:
\begin{equation}
A = \begin{bmatrix}
A_{11} & A_{12} \\
A_{21} & A_{22}
\end{bmatrix}
\end{equation}
\begin{equation}
B = \begin{bmatrix}
B_{11} & B_{12} \\
B_{21} & B_{22}
\end{bmatrix}
\end{equation}\\
Then we can write the matrix multiplication as follows:
\begin{equation}
AB = \begin{bmatrix}
A_{11}B_{11} + A_{12}B_{21} & A_{11}B_{12} + A_{12}B_{22} \\
A_{21}B_{11} + A_{22}B_{21} & A_{21}B_{12} + A_{22}B_{22}
\end{bmatrix}
\end{equation}\\
However, we still need to do $8$ matrix multiplications to get the result. The time complexity is $O(n^3)$.

\section{Strassen Algorithm}
The idea of Strassen algorithm is to reduce the number of matrix multiplications.
We can write the matrix multiplication as follows:
\begin{equation}
    M_1 = (A_{11} + A_{22})(B_{11} + B_{22})
\end{equation}
\begin{equation}
    M_2 = (A_{21} + A_{22})B_{11}
\end{equation}
\begin{equation}
    M_3 = A_{11}(B_{12} - B_{22})
\end{equation}
\begin{equation}
    M_4 = A_{22}(B_{21} - B_{11})
\end{equation}
\begin{equation}
    M_5 = (A_{11} + A_{12})B_{22}
\end{equation}
\begin{equation}
    M_6 = (A_{21} - A_{11})(B_{11} + B_{12})
\end{equation}
\begin{equation}
    M_7 = (A_{12} - A_{22})(B_{21} + B_{22})
\end{equation}\\
Then we can write the matrix multiplication $AB = C$ as follows:
\begin{equation}
    AB = C = \begin{bmatrix}
        C_{11} & C_{12} \\
        C_{21} & C_{22}
    \end{bmatrix}
\end{equation}
\begin{equation}
    C_{11} = M_1 + M_4 - M_5 + M_7
\end{equation}
\begin{equation}
    C_{12} = M_3 + M_5
\end{equation}
\begin{equation}
    C_{21} = M_2 + M_4
\end{equation}
\begin{equation}
    C_{22} = M_1 - M_2 + M_3 + M_6
\end{equation}\\
We can see that we only need to do $7$ matrix multiplications to get the result. The time complexity is $O(n^{\log_2 7})$.
In general, we will recursively do the matrix multiplication until the size of the matrix is $1 \times 1$.
Then we can use the basic matrix multiplication to get the result.

\end{document}